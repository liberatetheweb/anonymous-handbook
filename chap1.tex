\chapter{Введение}
\addcontentsline{toc}{part}{Введение}
\section{Причины появления книги}
В последнее время правительства многих стран стремятся уничтожить анонимность, оправдываясь <<безопасностью граждан>>. Однако данное стремление направлено не на увеличение безопасности, оно направлено только на усиление контроля. Чиновникам нужна уверенность в том, что на следующий день они не потеряют своего кресла, что вы проголосуете за них на следующих выборах, что вы не узнаете об их лжи, что вы не выйдете на улицы, недовольные их произволом, что они продолжат также воровать деньги из бюджета и брать взятки. Они ищут способ контроля. Государства, называющие себя демократическими, становятся авторитарными. Но правительства, вмешивающиеся в личную жизнь, не являются единственной причиной жажды анонимности. За вами может также следить ваш работодатель, хозяин хот-спота в любимой кафешке, администрация вашего учебного заведения. Маркетологи следят за вами, чтобы показать вам более таргетированную рекламу. Организации, борющиеся с пиратством, следят за вами, чтобы отсудить у вас деньги за две скачанные композиции уже умершего певца. И если вас все это не устраивает, то данная книга для вас.

\section{Связь с автором}
Вы можете сообщить об ошибках или связаться со мной по иным причинам с помощью электронной почты \href{mailto:anonhandbook@tormail.org}{anonhandbook@tormail.org} или отправив свои изменения в репозиторий \url{https://gitorious.org/anonymous-handbook}.
