\chapter{Введение}
\addcontentsline{toc}{part}{Введение}
\section{Причины появления книги}
В последнее время правительства многих стран стремятся уничтожить анонимность, оправдываясь <<безопасностью граждан>>. Политики с телеэкранов говорят нам, что преступники, пользуясь анонимностью, распространяют детскую порнографию, публикуют материалы, нарушающие авторские права. Что анонимностью придется пожертвовать для борьбы с терроризмом, с торговлей оружием и наркотиками. Но это абсолютная ложь. Подобно кухонному ножу, который может быть использован как и для убийства, так и для повседневной хозяйственной деятельности, анонимность может использоваться двойственно. Но подобный характер не является поводом для ее запрета или ограничения, как не является поводом для запрещения ножей возможность использовать их в насильственных действиях.

Государства, называющие себя демократическими, становятся авторитарными. Но правительства, вмешивающиеся в личную жизнь, не являются единственной причиной жажды анонимности. За вами может также следить ваш работодатель, хозяин хот-спота в любимой кафешке, администрация вашего учебного заведения. Маркетологи следят за вами, чтобы показать вам более таргетированную рекламу. Организации, борющиеся с пиратством, следят за вами, чтобы отсудить у вас деньги за две скачанные композиции уже умершего певца. И если вас все это не устраивает, то данная книга для вас.

\section{Связь с автором}
Вы можете сообщить об ошибках или связаться со мной по иным причинам с помощью электронной почты \href{mailto:anonhandbook@tormail.org}{anonhandbook@tormail.org} или отправив свои изменения в репозиторий \url{https://gitorious.org/anonymous-handbook}. Также доступен сайт \url{http://anonhandbook.org}, его зеркало в I2P \url{http://anonhandbook.i2p} и в Tor --- \url{http://oxgzwiiypou6udlp.onion}.
