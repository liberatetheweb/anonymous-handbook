\chapter{Анонимность в реальной жизни}
\section{Желтые точки}
\begin{wrapfigure}[9]{r}{0.25\linewidth}
\includegraphics[width=\linewidth]{dots}
\caption{Желтые точки. Изображение: Parhamr}
\end{wrapfigure}
При печати материалов (например, листовок) не стоит забывать, что многие принтеры  кодируют микроточками информацию о времени печати и о серийном номере принтера\cite{eff_dots}. Данная информация может быть использована для установления личности авторов отпечатков. Список принтеров, размещающих и не размещающих желтые точки смотрите в отчете Electronic Frontier Foundation\cite{eff_list}.
\section{Мобильные телефоны}
Использовать мобильные телефоны, в общем случае, не безопасно. Вышки сотовой связи имеют ограниченный радиус действия, а операторы связи знают, где находится каждая вышка и к какой вышке подключен в данным момент каждый абонент. Все SIM-карты на территории РФ должны оформляться на конкретного человека, личность которого подтверждается и заносится в специальную базу (однако, можно купить так называемую <<анонимную SIM-карту>>, такие SIM-карты продаются на различных форумах и аукционах). Также каждый мобильный телефон имеет уникальный номер --- IMEI, по которому можно определить владельца даже после смены SIM-карты (смена IMEI возможна, но не для всех моделей телефонов). Вся эта информация согласно законодательству РФ должна храниться в течении трех лет и предоставляться через систему СОРМ-3 органам федеральной безопасностих\cite{sorm_sorm3}.
\begin{enumerate}
\item Используйте анонимную SIM-карту.
\item Не используйте мобильный телефон в местах, которые можно однозначно связать с вами (работа, дом, образовательное учреждение и т.д.).
\end{enumerate}
