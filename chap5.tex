\chapter{Анонимность в реальной жизни}
\section{Желтые точки}
\begin{wrapfigure}[9]{r}{0.25\linewidth}
\includegraphics[width=\linewidth]{dots}
\caption{Желтые точки. Изображение: Parhamr}
\end{wrapfigure}
При печати материалов (например, листовок) не стоит забывать, что многие принтеры  кодируют микроточками информацию о времени печати и о серийном номере принтера\cite{eff_dots}. Данная информация может быть использована для установления  личности авторов отпечатков. Список принтеров, размещающих и не размещающих желтые точки смотрите в отчете Electronic Frontier Foundation\cite{eff_list}.
