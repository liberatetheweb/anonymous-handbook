\part{Необходимость анонимности в современном мире}
\chapter{Политические причины}
\section{Журналисты}
С 1993 по 2009 год в России было убито 176 журналистов\cite{kill}, .
\subsection{Избиение Михаила Бекетова}
Михаил Бекетов --- журналист, обладатель премии Press Freedom Award\cite{beketov_award}, премии правительства РФ в области печатных СМИ\cite{beketov_gosaward}, учредитель и главный редактор газеты <<Химкинская правда>>, в которой публиковались статьи с критикой в адрес химкинской администрации. 24 мая 2007 года была сожжена его машина\cite{beketov_car}, а 13 ноября 2008 года он был избит неизвестными\cite{beketov_beat}, после чего долгое время находился в больнице и получил инвалидность 1-й группы\cite{beketov_invalid}. Дело до сих пор не раскрыто.
\subsection{Нападения на Олега Кашина}
Олег Кашин --- российский полический журналист, неоднократно подвергавшийся нападениям. Наибольший резонанс получило нападение на него 6 ноября 2010 года. Двое неизвестных поджидали его около его дома, в котором он снимал квартиру. Место своего проживания он держал в тайне. Один из нападавших держал Олега Кашина, второй начал наносить удары железным прутом, спрятанным в букет\cite{kashin_beat}. Избиение продолжалось полторы минуты, за это время было нанесено 56 ударов\cite{kashin_count}. Журналиста доставили в больницу, где были диагностированы переломы нижних конечностей, лицевого скелета и черепно-мозговая травма\cite{kashin_trauma}. Несмотря на широкий общественный резонанс и поручение президента России Дмитрия Медведева о взятии дела под особый контроль\cite{kashin_medvedev}, оно до сих пор не раскрыто.
\subsection{Избиение Константина Фетисова}
\subsection{Убийство Анны Политковской}
\subsection{Убийство Натальи Эстемировой}
\subsection{Убийство Магомеда Евлоева}
\section{Гражданские активисты}