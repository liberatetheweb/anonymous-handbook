\part{Необходимость анонимности в современном мире}
\chapter{Политические причины}
\section{Журналисты}
По уровню свободы прессы, согласно отчету Freedom House 2012 года, Россия находится на 172 месте из 197 стран\cite{pressfreedom}, а по данным <<Репортеров без границ>> --- на 142 из 179\cite{rsf}. По уровню свободы слова в Интернете в 2011 году Россия занимала 22 место из 37\cite{netfreedom}. С 1993 по 2009 год в России было убито 176 журналистов\cite{kill}, а с 1998 года было совершено 871 нападение на журналистов и редакции\cite{attack}. С 2008 по 2011 год правозащитной организацией <<Агора>> было зафиксировано 23 попытки введения регулирования Интернета, 13 нападений на блогеров, 57 уголовных преследований, 30 гражданско-правовых санкций, 178 фактов административного давления, 239 случаев ограничения доступа\cite{agoranet,agoranet2011}.
\subsection{Избиение Михаила Бекетова}
Михаил Бекетов --- журналист, обладатель премии Press Freedom Award\cite{beketov_award}, премии правительства РФ в области печатных СМИ\cite{beketov_gosaward}, учредитель и главный редактор газеты <<Химкинская правда>>, в которой публиковались статьи с критикой в адрес химкинской администрации. 24 мая 2007 года была сожжена его машина\cite{beketov_car}, а 13 ноября 2008 года он был избит неизвестными\cite{beketov_beat}, после чего долгое время находился в больнице и получил инвалидность 1-й группы\cite{beketov_invalid}. Дело до сих пор не раскрыто.
\subsection{Нападения на Олега Кашина}
Олег Кашин --- российский полический журналист, неоднократно подвергавшийся нападениям. Наибольший резонанс получило нападение на него 6 ноября 2010 года. Двое неизвестных поджидали его около его дома, в котором он снимал квартиру. Место своего проживания он держал в тайне. Один из нападавших держал Кашина, второй начал наносить удары железным прутом, который он прятал в букете\cite{kashin_beat}. Избиение продолжалось полторы минуты, за это время было нанесено 56 ударов\cite{kashin_count}. Журналиста доставили в больницу, где были диагностированы переломы нижних конечностей, лицевого скелета и черепно-мозговая травма\cite{kashin_trauma}. Несмотря на широкий общественный резонанс и поручение президента России Дмитрия Медведева о взятии дела под особый контроль\cite{kashin_medvedev}, оно до сих пор не раскрыто.
\subsection{Убийство Анны Политковской}
Анна Политковская --- российская журналистка и правозащитница. Была застрелена около лифта своего дома 7 октября 2006 года\cite{politkovskaya_death}. 19 февраля 2009 года суд присяжных оправдал подозреваемых Ибрагима и Джабраила Махмудовых, Сергея Хаджикурбанова и Павла Рягузова\cite{politkovskaya_court1}, однако вскоре Верховный суд РФ отменил этот оправдательный приговор и отправил дело на новое рассмотрение\cite{politkovskaya_court2,politkovskaya_court3}. В августе 2008 года был задержан Дмитрий Павлюченков\cite{politkovskaya_pavluchenkov}. В марте 2012 года он заявил, что слежку за Политковской заказал Лом-Али Гайтукаев, получивший от Ахмеда Закаева заказ, который якобы был удобен Березовскому\cite{politkovskaya_head}. Следствие по делу все еще продолжается.
\subsection{Убийство Магомеда Евлоева}
Магомед Евлоев --- журналист, правозащитник, создатель сайта Ингушетия.Ru. Был убит выстрелом в голову 31 августа 2008 года. По официальной версии, Евлоев пытался отобрать автомат у сотрудника, сидевшего рядом с ним, после чего милиционер, находившийся рядом с водителем, выхватил пистолет и нацелил его на Евлоева. Выстрел, по утверждениям милиционеров, произошел случайно\cite{evloev_mvd}. По данным редакции портала Ингушетия.Ru, Евлоев прилетел в одном самолете с президентом республики Ингушетия Муратом Зязиковым. После того, как президент уехал, Евлоева окружили сотрудники охраны министра внутренних дел Ингушетии, силой посадили его в машину и увезли. По дороге они выстрелили ему в голову и выбросили из машины\cite{evloev_death}.
\subsection{Убийство Пола Хлебникова}
Пол Хлебников --- публицист, журналист, главный редактор русскоязычной редакции журнала Forbes. 9 июля 2004 года был застрелен около офиса российского отделения Forbes в Москве\cite{hlebnikov_death}. Дело так и не было раскрыто.
\section{Гражданские активисты}
\section{Блогеры}
\subsection{Дело Саввы Терентьева}
\subsection{Дело Георгия Саркисяна}
\subsection{Дело Ирека Муртазина}
\subsection{Дело Ивана Перегородиева}
\subsection{Газпромбанк против блогеров}