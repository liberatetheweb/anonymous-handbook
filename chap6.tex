\chapter{Законы, ограничивающие свободу слова и анонимность}
\section{Постановление Правительства РФ от 16 апреля 2012 г. № 313}
\textbf{Постановление Правительства РФ от 16 апреля 2012 г. № 313 <<Об утверждении Положения о лицензировании деятельности по разработке, производству, распространению шифровальных (криптографических) средств, информационных систем и телекоммуникационных систем, защищенных с использованием шифровальных (криптографических) средств, выполнению работ, оказанию услуг в области шифрования информации, техническому обслуживанию шифровальных (криптографических) средств, информационных систем и телекоммуникационных систем, защищенных с использованием шифровальных (криптографических) средств (за исключением случая, если техническое обслуживание шифровальных (криптографических) средств, информационных систем и телекоммуникационных систем, защищенных с использованием шифровальных (криптографических) средств, осуществляется для обеспечения собственных нужд юридического лица или индивидуального предпринимателя)>>} запрещает практически любую деятельность, связанную с криптографией, за исключением деятельности с использованием:
\begin{enumerate}
\item шифровальных (криптографических) средств, предназначенных для защиты информации, содержащей сведения, составляющие государственную тайну;
\item шифровальных (криптографических) средств, а также товаров, содержащих шифровальные (криптографические) средства, реализующих либо симметричный криптографический алгоритм, использующий криптографический ключ длиной, не превышающей 56  бит, либо ассиметричный криптографический алгоритм, основанный либо на методе разложения на множители целых чисел, размер которых не превышает 512  бит, либо на методе вычисления дискретных логарифмов в мультипликативной группе конечного поля размера, не превышающего 512  бит, либо на методе вычисления дискретных логарифмов в иной группе размера, не превышающего 112  бит;
\item товаров, содержащих шифровальные (криптографические) средства, имеющих либо функцию аутентификации, включающей в себя все аспекты контроля доступа, где нет шифрования файлов или текстов, за исключением шифрования, которое непосредственно связано с защитой паролей, персональных идентификационных номеров или подобных данных для защиты от несанкционированного доступа, либо имеющих электронную подпись;
\item шифровальных (криптографических) средств, являющихся компонентами программных операционных систем, криптографические возможности которых не могут быть изменены пользователями, которые разработаны для установки пользователем самостоятельно без дальнейшей существенной поддержки поставщиком и техническая документация (описание алгоритмов криптографических преобразований, протоколы взаимодействия, описание интерфейсов и т.д.) на которые является доступной;
\item персональных смарт-карт (интеллектуальных карт), криптографические возможности которых ограничены использованием в оборудовании или системах, указанных в подпунктах »е» - »и» настоящего пункта, или персональных смарт-карт (интеллектуальных карт) для широкого общедоступного применения, криптографические возможности которых недоступны пользователю и которые в результате специальной разработки имеют ограниченные возможности защиты хранящейся на них персональной информации;
\item приемной аппаратуры для радиовещания, коммерческого телевидения или аналогичной коммерческой аппаратуры для вещания на ограниченную аудиторию без шифрования цифрового сигнала, кроме случаев использования шифрования исключительно для управления видео- или аудиоканалами и отправки счетов или возврата информации, связанной с программой, провайдерам вещания;
\item оборудования, криптографические возможности которого недоступны пользователю, специально разработанного и ограниченного для осуществления следующих функций:
    \begin{enumerate}
    \item исполнение программного обеспечения в защищенном от копирования виде;
    \item обеспечение доступа к защищенному от копирования содержимому, хранящемуся только на доступном для чтения носителе информации, либо доступа к информации, хранящейся в зашифрованной форме на носителях, когда эти носители информации предлагаются на продажу населению в идентичных наборах;
    \item контроль копирования аудио- и видеоинформации, защищенной авторскими правами;
    \end{enumerate}
\item шифровального (криптографического) оборудования, специально разработанного и ограниченного применением для банковских или финансовых операций в составе терминалов единичной продажи (банкоматов), POS-терминалов и терминалов оплаты различного вида услуг, криптографические возможности которых не могут быть изменены пользователями;
\item портативных или мобильных радиоэлектронных средств гражданского назначения (например, для использования в коммерческих гражданских системах сотовой радиосвязи), которые не способны к сквозному шифрованию (то есть от абонента к абоненту);
\item беспроводного оборудования, осуществляющего шифрование информации только в радиоканале с максимальной дальностью беспроводного действия без усиления и ретрансляции менее 400 м в соответствии с техническими условиями производителя (за исключением оборудования, используемого на критически важных объектах);
\item шифровальных (криптографических) средств, используемых для защиты технологических каналов информационно-телекоммуникационных систем и сетей связи, не относящихся к критически важным объектам;
\item товаров, у которых криптографическая функция гарантированно заблокирована производителем.
\end{enumerate}
Полный текст постановления и приложения к нему можно прочитать здесь: \url{http://government.ru/gov/results/18742/}.
\section{Указ Президента РФ от 3 апреля 1995 № 334}
Указ запрещает деятельность юридических и физических лиц, связанную с разработкой, производством, реализацией и эксплуатацией шифровальных средств, а также защищенных технических средств хранения, обработки и передачи информации, предоставлением услуг в области шифрования информации без лицензий, а также ввоз шифровальных средств иностранного производства без лицензии Министерства внешних экономических связей Российской Федерации\cite{334}.
\section{Федеральный закон Российской Федерации от 28 июля 2012 г. № 139-ФЗ}
\textbf{Федеральный закон Российской Федерации от 28 июля 2012 г. № 139-ФЗ <<О внесении изменений в Федеральный закон "О защите детей от информации, причиняющей вред их здоровью и развитию" и отдельные законодательные акты Российской Федерации>>}, также известный как \textbf{Законопроект № 89417-6} --- закон, установший цензуру в российском сегменте сети Интернет. Согласно закону, в России создается <<Единый реестр доменных имен и (или) универсальных указателей страниц сайтов в сети Интернет и сетевых адресов сайтов в сети Интернет, содержащих информацию, запрещенную к распространению на территории Российской Федерации федеральными законами>>, в который вносятся доменные имена или ссылки на страницы. В течении суток о внесении в список должен быть проинформирован владелец сайта, если владелец не удалит информацию, из-за которой страница попала в реестр, то хостинг-провайдер обязан заблокировать сайт, тоже в течении суток. Если и он это не делает, то доступ к сайту обязаны ограничить операторы связи.\\
Против данного закона высказались: Совет по правам человека при президенте РФ\cite{139_presidentsoviet}, Русская Википедия\cite{139_wiki}, Яндекс\cite{139_yandex}, Google\cite{139_google}, LiveJournal\cite{139_livejournal}, Вконтакте\cite{139_vk}.\\
Законопроект был принят Государственной Думой Российской Федерации 11 июля 2012 года во втором и третьем чтении\cite{139_gosduma}.\\
28 июля законопроект был подписан президентом РФ Владимиром Путиным, а 30 июля 2012 года был опубликован и вступил в силу\cite{139_rg,139_blacklist}.\\
Согласно Постановлению Правительства Российской Федерации от 26 октября 2012 года № 1101, организацией, составляющей список запрещенных сайтов, стал Роскомнадзор\cite{139_1101}.\\
1 ноября 2012 года на сайте \url{http://zapret-info.gov.ru} появилась форма для проверки наличия сайта в черном списке, а также форма для жалобы на контент в сети. Полностью список не публикуется.\\
Полный текст федерального закона: \url{http://rg.ru/2012/07/30/zakon-dok.html}.
\section{СОРМ}
\textbf{Система технических средств для обеспечения функций оперативно-розыскных мероприятий (СОРМ)} --- комплекс технических средства, направленных на обеспечение возможности проведения оперативно-розыскных мероприятий в сетях телефонной, подвижной и беспроводной связи и сетях персонального радиовызова общего пользования.\\
Поставщики услуг связи обязаны устанавливать СОРМ за свои деньги.\\
Следует различать СОРМ-1, созданный для прослушивания телефонных разговоров, СОРМ-2, созданный для протоколизации Интернет-соединений\cite{sorm_12} и СОРМ-3 --- комплекс, предназначенный для сбора, хранения и обработки информации об абонентах и оказанных им услугам связи, а также для предоставления оперативного доступа к этим данным\cite{sorm_3}. В нормативно-правовых актах встречается только термин <<СОРМ>>, термины <<СОРМ-2>> и <<СОРМ-3>> являются условными\cite{sorm_name}.
\subsection{СОРМ-1}
\textbf{Система технических средств по обеспечению оперативно-розыскных мероприятий на сетях подвижной радиотелефонной связи (СОРМ СПРС)} --- система прослушивания телефонных разговоров и установления местоположения абонентов.
СОРМ СПРС должна обеспечивать\cite{sorm_sorm1}:
\begin{enumerate}
\item контроль исходящих и входящих вызовов контролируемых подвижных абонентов в СПРС;
\item контроль исходящих вызовов (местных, внутризоновых, междугородных и международных) от всех абонентов СПРС к определенным абонентам (анализ по номеру В);
\item предоставление данных о местоположении контролируемых подвижных абонентов (ПА), подвижных станций (ПС) при их перемещении по СПРС;
\item сохранение контроля за установленным соединением при процедурах передачи управления вызовом (handover) как между базовыми станциями (БС) в пределах одного центра коммутации подвижной связи (ЦКП), так и разных ЦКП;
\item контроль вызовов при предоставлении ПА дополнительных услуг связи, в частности, изменяющих направление вызова (Call Forwarding). При предоставлении ПА такой услуги в процессе установления соединения должны контролироваться номера, на которые вызов перенаправляется (возможно неоднократное перенаправление вызова до установления разговорного состояния);
\item контроль за соединениями, обеспечивающими передачу телефонной и нетелефонной информации (передача данных, факсимильная связь, короткие сообщения);
\item при предоставлении контролируемому ПА дополнительной услуги, обеспечивающей возможность ПА одновременного разговора с несколькими абонентами, например <<конференцсвязь>>, должны контролироваться номера всех абонентов;
\item возможность получения по запросу с пункта управления (ПУ) информации о ПА по его идентификатору или присвоенному номеру телефонной сети общего пользования (ТфОП), цифровой сети с интеграцией служб (ЦСИС), а именно предоставляемые данному ПА услуги связи.
\end{enumerate}
\subsection{СОРМ-2}
\textbf{СОРМ-2} --- комплекс технических средств, направленных на осущественние оперативно-розыскных мероприятий путем логгирования и перехвата Интернет-трафика.\\
Сеть передачи данных обеспечивает техническую возможность передачи на пункт управления ОРМ следующей информации, относящейся к контролируемым соединениям и (или) сообщениям электросвязи, в процессе установления соединений и (или) передачи сообщений электросвязи\cite{sorm_sorm2}:
\begin{enumerate}
\item о выделенных абоненту (пользователю) сетевых адресах (IP-адресах) до реализации функции преобразования (трансляции) сетевых адресов и до начала передачи первого информационного пакета, а также информации о завершении контролируемого соединения;
\item передаваемой в контролируемом соединении и (или) сообщении электросвязи, включая информацию, связанную с обеспечением процесса оказания услуг связи в том виде и последовательности, в которых такая информация поступала с пользовательского (оконечного) оборудования или из присоединенной сети связи;
\item о местоположении пользовательского (оконечного) оборудования, используемого для передачи (приема) информации контролируемого соединения и (или) сообщения электросвязи, за исключением сетей передачи данных, в которых не предусмотрена технологическая возможность определения местоположения пользовательского (оконечного) оборудования.\end{enumerate}
\subsection{СОРМ-3}
\textbf{СОРМ-3} --- комплекс технических средств, предназначенный для сбора, хранения и обработки информации об абонентах и оказанных им услугам связи, а также для предоставления оперативного доступа к этим данным.\\
Оператор связи обязан своевременно обновлять информацию, содержащуюся в базах данных об абонентах оператора связи и оказанных им услугах связи.\\
Базы данных должны содержать следующую информацию об абонентах оператора связи:
\begin{enumerate}
\item фамилия, имя, отчество, место жительства и реквизиты основного документа, удостоверяющего личность, представленные при личном предъявлении абонентом указанного документа, --- для абонента-гражданина;
\item наименование (фирменное наименование) юридического лица, его место нахождения, а также список лиц, использующих оконечное оборудование юридического лица, заверенный уполномоченным представителем юридического лица, в котором указаны их фамилии, имена, отчества, места жительства и реквизиты основного документа, удостоверяющего личность, --- для абонента-юридического лица;
\item сведения баз данных о расчетах за оказанные услуги связи, в том числе о соединениях, трафике и платежах абонентов.
\end{enumerate}
Указанная информация должна храниться оператором связи в течение 3 лет и предоставляться органам федеральной службы безопасности, а в случае отсутствия у органов федеральной службы безопасности необходимых оперативно-технических возможностей для проведения оперативно-разыскных мероприятий, связанных с использованием технических средств, указанные мероприятия осуществляют органы внутренних дел, являющиеся уполномоченными органами, в том числе в интересах других уполномоченных органов, путем осуществления круглосуточного удаленного доступа к базам данных\cite{sorm_sorm3}.
