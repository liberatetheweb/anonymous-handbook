\chapter{Необходимость анонимности в современном мире}
\section{Преследования}
\subsection{Обычные люди}
\subsubsection{Попадание данных о платежах <<РосПилу>> в руки активистам движения <<Наши>>}
РосПил --- некоммерческий проект <<Фонда борьбы с коррупцией>>, занимающийся мониторингом государственных закупок с целью установления фактов коррупции. Финансируется за счет добровольных пожертвований.\\
Движение <<Наши>> --- прокремлевское молодежное движение.\\
В апреле и мае 2011 года от имени некоторых новостных изданий неизвестными лицами были совершены звонки людям, которые отправляли пожертвования проекту <<РосПил>>\cite{rospil_call}. Переводы осуществлялись на кошелек в системе Яндекс.Деньги. Представители Яндекса отметили, что данные по сотне человек, переводивших деньги на счет РосПила они действительно предоставляли ФСБ РФ\cite{rospil_fsb}. Позже было выяснено, что звонки осуществляла комиссар движения <<Наши>> Юлия Дихтяр\cite{rospil_nashi}. В ФСБ отказались комментировать то, как данные, предоставленные им, получили третьи лица\cite{rospil_nashi}.
\subsubsection{Дело Витольда Филиппова}
Витольд Филиппов поставил <<лайк>> под кадром из фильма <<Американская история Икс>>, не запрещенного ни в одной стране. Прокуратура усмотрела в этом случай распространения нацистской символики\cite{vitold_like}. 24 августа 2012 года он был приговорен к штрафу в 1000 рублей по статье 20.3 КоАП РФ (<<Пропаганда и публичное демонстрирование нацистской атрибутики или символики>>) без права обжалования\cite{vitold_judge}.
\subsection{Журналисты}
По уровню свободы прессы, согласно отчету Freedom House 2012 года, Россия находится на 172 месте из 197 стран\cite{pressfreedom}, а по данным <<Репортеров без границ>> --- на 142 из 179\cite{rsf}. С 1993 по 2009 год в России было убито 176 журналистов\cite{kill}, а с 1998 года было совершено 871 нападение на журналистов и редакции\cite{attack}.
\subsubsection{Избиение Михаила Бекетова}
Михаил Бекетов --- журналист, обладатель премии Press Freedom Award\cite{beketov_award}, премии правительства РФ в области печатных СМИ\cite{beketov_gosaward}, учредитель и главный редактор газеты <<Химкинская правда>>, в которой публиковались статьи с критикой в адрес химкинской администрации. 24 мая 2007 года была сожжена его машина\cite{beketov_car}, а 13 ноября 2008 года он был избит неизвестными\cite{beketov_beat}, после чего долгое время находился в больнице и получил инвалидность 1-й группы\cite{beketov_invalid}. Дело до сих пор не раскрыто.
\subsubsection{Нападения на Олега Кашина}
Олег Кашин --- российский полический журналист, неоднократно подвергавшийся нападениям. Наибольший резонанс получило нападение на него 6 ноября 2010 года. Двое неизвестных поджидали его около его дома, в котором он снимал квартиру. Место своего проживания он держал в тайне. Один из нападавших держал Кашина, второй начал наносить удары железным прутом, который он прятал в букете\cite{kashin_beat}. Избиение продолжалось полторы минуты, за это время было нанесено 56 ударов\cite{kashin_count}. Журналиста доставили в больницу, где были диагностированы переломы нижних конечностей, лицевого скелета и черепно-мозговая травма\cite{kashin_trauma}. Несмотря на широкий общественный резонанс и поручение президента России Дмитрия Медведева о взятии дела под особый контроль\cite{kashin_medvedev}, оно до сих пор не раскрыто.
\subsubsection{Убийство Анны Политковской}
Анна Политковская --- российская журналистка и правозащитница. Была застрелена около лифта своего дома 7 октября 2006 года\cite{politkovskaya_death}. 19 февраля 2009 года суд присяжных оправдал подозреваемых Ибрагима и Джабраила Махмудовых, Сергея Хаджикурбанова и Павла Рягузова\cite{politkovskaya_court1}, однако вскоре Верховный суд РФ отменил этот оправдательный приговор и отправил дело на новое рассмотрение\cite{politkovskaya_court2,politkovskaya_court3}. В августе 2008 года был задержан Дмитрий Павлюченков\cite{politkovskaya_pavluchenkov}. В марте 2012 года он заявил, что слежку за Политковской заказал Лом-Али Гайтукаев, получивший от Ахмеда Закаева заказ, который якобы был удобен Березовскому\cite{politkovskaya_head}. Следствие по делу все еще продолжается.
\subsubsection{Убийство Магомеда Евлоева}
Магомед Евлоев --- журналист, правозащитник, создатель сайта Ингушетия.Ru. Был убит выстрелом в голову 31 августа 2008 года. По официальной версии, Евлоев пытался отобрать автомат у сотрудника, сидевшего рядом с ним, после чего милиционер, находившийся рядом с водителем, выхватил пистолет и нацелил его на Евлоева. Выстрел, по утверждениям милиционеров, произошел случайно\cite{evloev_mvd}. По данным редакции портала Ингушетия.Ru, Евлоев прилетел в одном самолете с президентом республики Ингушетия Муратом Зязиковым. После того, как президент уехал, Евлоева окружили сотрудники охраны министра внутренних дел Ингушетии, силой посадили его в машину и увезли. По дороге они выстрелили ему в голову и выбросили из машины\cite{evloev_death}.
\subsubsection{Убийство Пола Хлебникова}
Пол Хлебников --- публицист, журналист, главный редактор русскоязычной редакции журнала Forbes. 9 июля 2004 года был застрелен около офиса российского отделения Forbes в Москве\cite{hlebnikov_death}. Дело так и не было раскрыто.
\subsection{Блогеры}
С 2008 по 2011 год правозащитной организацией <<Агора>> было зафиксировано 23 попытки введения регулирования Интернета, 13 нападений на блогеров, 57 уголовных преследований, 30 гражданско-правовых санкций, 178 фактов административного давления, 239 случаев ограничения доступа\cite{agoranet,agoranet2011}. По уровню свободы слова в Интернете в 2011 году Россия занимала 22 место из 37\cite{netfreedom}.
\subsubsection{Дело Саввы Терентьева}
Савва Терентьев --- блогер из Сыктывкара, фигурант первого уголовного дела в России, возбужденного за комментарий в блоге. Комментарий был оставлен 15 февраля 2007 года к записи, повествующей об изъятии жестких дисков компьютеров, принадлежащих редакции газеты <<Искра>> сотрудниками отдела <<К>>\cite{terentyev_k}.
Текст комментария, оставленного блогером\cite{terentyev_quote}:
\begin{quote}ненавижу ментов сцуконах. не согласен с тезисом "у милиционеров остался менталитет репрессивной дубинки в руках властьимущих", во-первых, у ментов, во-вторых, не остался. он просто-напросто неискореним. мусор - и в африке мусор. кто идет в менты - быдло, гопота - самые тупые, необразованные представители жив(отн)ого мира. было бы хорошо, если б в центре каждого города россии, на главной площади (в сыктывкаре - прям в центре стефановской, где елка стоит- чтоб всем видно было) стояла печь, как в освенциме, где церемониально, ежедневно, а лучше дважды в сутки (в полдень и полночь например) - сжигали бы по неверному менту, народ, чтоб сжигал, это был бы первый шаг к очищению общества от ментовско-гопотской грязи\end{quote}
7 июля 2008 года Терентьев получил год условно\cite{terentyev_year}. 12 июля 2011 года он получил политическое убежище в Эстонии\cite{terentyev_estonia}.
\subsubsection{Дело Ирека Муртазина}
Ирек Муртазин --- журналист, блогер, автор книги <<Минтимер Шаймиев: последний президент Татарстана>> и политический активист. 12 сентября 2008 года он разместил в своем блоге информацию о том, что президент Республики Татарстан, Минтимер Шаймиев, скончался. Текст размещенного поста выглядел так\cite{murtazin_post}:
\begin{quote}
Пришла страшная весть...\\
...на 72-ом году жизни, во время отдыха в Турции (в  Кемере) скоропостижно скончался Минтимер Шарипович Шаймиев.
Честно говоря – не верится. Точнее, не хочется верить. Потому что, если это правда, то начнется такая свара, такая нешуточная борьба за то, чтобы занять освободившееся кресло, что чубы у холопов будут трещать и вдоль и поперек. И именно из-за подобных перспектив, ближайшее окружение Минтимера Шариповича попытается скрыть эту информацию. Чтобы успеть перегруппироваться (вплоть до скоропостижной эвакуации из страны). Именно поэтому официальная информация, думаю, будет не раньше чем через неделю
\end{quote}
Муртазин был не первым человеком, разместившим информацию о смерти президента. Например, за два часа до его поста, появился пост в сообществе kazan с вопросом об истинности слухов о смерти Шаймиева\cite{murtazin_later}. Пресс-служба Шаймиева быстро опровергла слухи о смерти президента\cite{murtazin_alive}. 10 декабря 2008 на блогера было заведено уголовное дело по обвинению в клевете и нарушении неприкосновенности частной жизни\cite{murtazin_delo}. 29 декабря 2009 года на него напали неизвестные и избили\cite{murtazin_beat}. Сам Муртазин считает нападение на него политическим\cite{murtazin_political}. 26 ноября 2009 года Ирека Муртазина признали виновным в клевете и возбуждение социальной ненависти либо вражды к определенной социальной группе (представителям власти\cite{murtazin_group}) с угрозой применения насилия и назначили наказание в виде 1 года и 9 месяцев лишения свободы с отбыванием наказания в колонии-поселении\cite{murtazin_court}. 31 января 2011 года было удовлетворено заявление об условно-досрочном освобождении Муртазина\cite{murtazin_free}.
