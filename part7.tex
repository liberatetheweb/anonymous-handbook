\chapter{Законы, ограничивающие свободу слова и анонимность}
\section{Постановление Правительства РФ от 16 апреля 2012 г. №313}
{Постановление Правительства РФ от 16 апреля 2012 г. №313 <<Об утверждении Положения о лицензировании деятельности по разработке, производству, распространению шифровальных (криптографических) средств, информационных систем и телекоммуникационных систем, защищенных с использованием шифровальных (криптографических) средств, выполнению работ, оказанию услуг в области шифрования информации, техническому обслуживанию шифровальных (криптографических) средств, информационных систем и телекоммуникационных систем, защищенных с использованием шифровальных (криптографических) средств (за исключением случая, если техническое обслуживание шифровальных (криптографических) средств, информационных систем и телекоммуникационных систем, защищенных с использованием шифровальных (криптографических) средств, осуществляется для обеспечения собственных нужд юридического лица или индивидуального предпринимателя)>> запрещает практически любую деятельность, связанную с криптографией, за исключением деятельности с использованием:
\begin{enumerate}
\item шифровальных (криптографических) средств, предназначенных для защиты информации, содержащей сведения, составляющие государственную тайну;
\item шифровальных (криптографических) средств, а также товаров, содержащих шифровальные (криптографические) средства, реализующих либо симметричный криптографический алгоритм, использующий криптографический ключ длиной, не превышающей 56  бит, либо ассиметричный криптографический алгоритм, основанный либо на методе разложения на множители целых чисел, размер которых не превышает 512  бит, либо на методе вычисления дискретных логарифмов в мультипликативной группе конечного поля размера, не превышающего 512  бит, либо на методе вычисления дискретных логарифмов в иной группе размера, не превышающего 112  бит;
\item товаров, содержащих шифровальные (криптографические) средства, имеющих либо функцию аутентификации, включающей в себя все аспекты контроля доступа, где нет шифрования файлов или текстов, за исключением шифрования, которое непосредственно связано с защитой паролей, персональных идентификационных номеров или подобных данных для защиты от несанкционированного доступа, либо имеющих электронную подпись;
\item шифровальных (криптографических) средств, являющихся компонентами программных операционных систем, криптографические возможности которых не могут быть изменены пользователями, которые разработаны для установки пользователем самостоятельно без дальнейшей существенной поддержки поставщиком и техническая документация (описание алгоритмов криптографических преобразований, протоколы взаимодействия, описание интерфейсов и т.д.) на которые является доступной;
\item персональных смарт-карт (интеллектуальных карт), криптографические возможности которых ограничены использованием в оборудовании или системах, указанных в подпунктах »е» - »и» настоящего пункта, или персональных смарт-карт (интеллектуальных карт) для широкого общедоступного применения, криптографические возможности которых недоступны пользователю и которые в результате специальной разработки имеют ограниченные возможности защиты хранящейся на них персональной информации;
\item приемной аппаратуры для радиовещания, коммерческого телевидения или аналогичной коммерческой аппаратуры для вещания на ограниченную аудиторию без шифрования цифрового сигнала, кроме случаев использования шифрования исключительно для управления видео- или аудиоканалами и отправки счетов или возврата информации, связанной с программой, провайдерам вещания;
\item оборудования, криптографические возможности которого недоступны пользователю, специально разработанного и ограниченного для осуществления следующих функций:
    \begin{enumerate}
    \item исполнение программного обеспечения в защищенном от копирования виде;
    \item обеспечение доступа к защищенному от копирования содержимому, хранящемуся только на доступном для чтения носителе информации, либо доступа к информации, хранящейся в зашифрованной форме на носителях, когда эти носители информации предлагаются на продажу населению в идентичных наборах;
    \item контроль копирования аудио- и видеоинформации, защищенной авторскими правами;
    \end{enumerate}
\item шифровального (криптографического) оборудования, специально разработанного и ограниченного применением для банковских или финансовых операций в составе терминалов единичной продажи (банкоматов), POS-терминалов и терминалов оплаты различного вида услуг, криптографические возможности которых не могут быть изменены пользователями;
\item портативных или мобильных радиоэлектронных средств гражданского назначения (например, для использования в коммерческих гражданских системах сотовой радиосвязи), которые не способны к сквозному шифрованию (то есть от абонента к абоненту);
\item беспроводного оборудования, осуществляющего шифрование информации только в радиоканале с максимальной дальностью беспроводного действия без усиления и ретрансляции менее 400 м в соответствии с техническими условиями производителя (за исключением оборудования, используемого на критически важных объектах);
\item шифровальных (криптографических) средств, используемых для защиты технологических каналов информационно-телекоммуникационных систем и сетей связи, не относящихся к критически важным объектам;
\item товаров, у которых криптографическая функция гарантированно заблокирована производителем.
\end{enumerate}
Полный текст постановления и приложения к нему можно прочитать здесь: \url{http://government.ru/gov/results/18742/}.
\section{Указ Президента РФ от 3 апреля 1995 №334}
\section{Законопроект № 89417-6}
Законопроект, вносящий изменения в Федеральный закон от 29 декабря 2010 года № 436-ФЗ <<О защите детей от информации, причиняющей вред их здоровью и развитию>>, Кодекс Российской Федерации об административных правонарушениях, Федеральный закон от 7 июля 2003 г. № 126-ФЗ <<О связи>> и Федеральный закон от 27 июля 2006 г. № 149-ФЗ <<Об информации, информационных технологиях и о защите информации>>.\\
Законопроект устанавливает цензуру в российском сегменте сети Интернет. Создается <<Единый реестр доменных имен и (или) универсальных указателей страниц сайтов в сети Интернет и сетевых адресов сайтов в сети Интернет, содержащих информацию, запрещенную к распространению на территории Российской Федерации федеральными законами>>, в который вносятся доменные имена или ссылки на страницы. В течении суток об этом должен быть проинформирован владелец сайта, если владелец не удалит информацию, из-за которой страница попала в реестр, то хостинг-провайдер обязан заблокировать сайт. Если и он это не делает, то доступ к сайту обязаны ограничить операторы связи.\\
Против данного законопроекта высказалась Русская Википедия\cite{89417-6_wiki}, Яндекс\cite{89417-6_yandex}, Google\cite{89417-6_google}, LiveJournal\cite{89417-6_livejournal}, Вконтакте\cite{89417-6_vk}.\\
Законопроект был принят Государственной Думой Российской Федерации 11 июля 2012 года во втором и третьем чтении\cite{89417-6_gosduma}.
Полный текст законопроекта можно прочитать здесь --- \url{http://asozd2.duma.gov.ru/main.nsf/%28Spravka%29?OpenAgent&RN=89417-6}.
