\chapter{Законы, гарантирующие свободу слова и анонимность}
\section{Статья 23 Конституции РФ}
\begin{enumerate}
\item Каждый имеет право на неприкосновенность частной жизни, личную и семейную тайну, защиту своей чести и доброго имени.
\item Каждый имеет право на тайну переписки, телефонных переговоров, почтовых, телеграфных и иных сообщений. Ограничение этого права допускается только на основании судебного решения.
\end{enumerate}
\section{Статья 24 Конституции РФ}
\begin{enumerate}
\item Сбор, хранение, использование и распространение информации о частной жизни лица без его согласия не допускаются.
\item Органы государственной власти и органы местного самоуправления, их должностные лица обязаны обеспечить каждому возможность ознакомления с документами и материалами, непосредственно затрагивающими его права и свободы, если иное не предусмотрено законом.
\end{enumerate}
\section{Статья 29 Конституции РФ}
\begin{enumerate}
\item Каждому гарантируется свобода мысли и слова.
\item Не допускаются пропаганда или агитация, возбуждающие социальную, расовую, национальную или религиозную ненависть и вражду. Запрещается пропаганда социального, расового, национального, религиозного или языкового превосходства.
\item Никто не может быть принужден к выражению своих мнений и убеждений или отказу от них.
\item Каждый имеет право свободно искать, получать, передавать, производить и распространять информацию любым законным способом. Перечень сведений, составляющих государственную тайну, определяется федеральным законом.
\item Гарантируется свобода массовой информации. Цензура запрещается.
\end{enumerate}