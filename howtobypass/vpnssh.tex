\begin{center}
\LARGE \textbf{Обходим Великий Российский Файрвол}\\
\large Часть 3 из 5\\
\normalsize\textbf{В России официально установлена цензура Интернета. Но контролировать Интернет не так просто, как это может показаться с первого взгляда.}
\end{center}
\textbf{VPN} --- технология, позволяющая создавать сети поверх существующего Интернет-подключения. Из-за высокой скорости работы, простоты настройки и шифрования трафика от клиента до VPN-провайдера часто используется как средство сокрытия реального IP-адреса при доступе в Интернет. VPN-провайдеры обычно предоставляют свои услуги на платной основе.\\
\textbf{SSH} --- протокол, созданный для безопасной передачи данных. Часто используется для удаленного управления другими компьютерами, но может использоваться и для создания туннелей.\\
\textbf{SSH-туннель} --- туннель, созданный с помощью SSH-соединения и используемый для передачи данных. Существуют организации, предоставляющие SSH-туннелирование на платной основе.
\subsection*{Использование}
\begin{important}
Не забывайте, что хозяин VPN-сервиса или SSH-туннеля видит трафик нешифрованным.
\end{important}
При использовании VPN, следуйте инструкциям, полученным от вашего VPN-провайдера.\\
SSH-туннели настраиваются так:
\begin{lstlisting}
ssh -D localhost:port login@address
\end{lstlisting}
port --- порт, трафик на который будет пропускаться через SSH-туннель.\\
login --- ваш логин на удаленном сервере.\\
address --- адрес удаленного сервера.\\
После этого установите в приложениях, трафик которых вы хотите туннелировать, например, в браузере, адрес SOCKS-прокси localhost с портом, который вы указали в предыдущем шаге.
\subsection*{Некоторые VPN-провайдеры}
\begin{enumerate}
\item \url{https://ipredator.se}
\item \url{https://kebrum.com}
\item \url{https://relakks.com}
\item \url{https://vpntunnel.se}
\item \url{http://ivacy.com}
\end{enumerate}
\subsection*{Некоторые провайдеры SSH-туннелей}
\begin{enumerate}
\item \url{https://tunnelr.com}
\item \url{http://torvpn.com}
\item \url{http://vpnsecure.me}
\item \url{http://guardster.com}
\item \url{http://anonyproz.com}
\end{enumerate}
\vfill
\scriptsize Создано на основе материалов из <<Настольной книги анонима>> --- \url{anonhandbook.org}, \url{anonhandbook.i2p}, \url{oxgzwiiypou6udlp.onion}
\normalsize
