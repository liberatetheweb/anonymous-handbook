\begin{center}
\LARGE \textbf{Обходим Великий Российский Файрвол}\\
\large Часть 4 из 5\\
\normalsize\textbf{В России официально установлена цензура Интернета. Но контролировать Интернет не так просто, как это может показаться с первого взгляда.}
\end{center}
\textbf{JonDo} (\textbf{JonDonym}, также \textbf{Java Anon Proxy} или \textbf{JAP}) --- программное обеспечение, представляющее доступ к цепочке прокси-серверов.
\subsection*{Установка}
Для установки посетите \url{https://anonymous-proxy-servers.net} или установите пакет с помощью пакетного менеджера вашего дистрибутива.
\subsection*{Использование}
\begin{important}
Не забывайте, что хозяева выходных нод видят трафик нешифрованным.
\end{important}
JonDo напоминает Tor, но в отличии от Tor, где каждый доброволец может поднять как промежуточный сервер, так и exit-ноду, JonDo опирается на помощь отдельных организаций. Однако, Tor может использоваться в цепочке JonDo, для этого достаточно добавить адрес socks5-прокси Tor.\\
Бесплатная версия позволяет проксировать только HTTP и HTTPS трафик, в платной версии доступно все, а также нелимитирована скорость.\\
Для использования запустите JonDo и настройте браузер на использование прокси-сервера\\127.0.0.1:4001.
\subsection*{Недостатки}
\begin{enumerate}
\item В бесплатной версии можно проксировать только HTTP и HTTPS.
\item В бесплатной версии скорость ограничена до 30--50 кБит/с.
\item В бесплатной версии размер передаваемого файла ограничен 2 МБ.
\item Число нод очень сильно ограничено.
\end{enumerate}

\vfill
\scriptsize Создано на основе материалов из <<Настольной книги анонима>> --- \url{anonhandbook.org}, \url{anonhandbook.i2p}, \url{oxgzwiiypou6udlp.onion}
\normalsize
