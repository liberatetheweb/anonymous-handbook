\chapter{Общие правила анонимности}
\section{Нераскрытие информации}
Даже если информация кажется вам незначимой и вы считаете, что по ней будет сложно установить вашу личность, не раскрывайте ее. Информация об одном лишь поле позволяет сократить число вариантов примерно в два раза. Комбинируя известную информацию, легко можно определить конкретного человека. Не распространяйтесь ни о своем роде занятий, ни о музыкальных вкусах, ни тем более о месте жительства и возрасте, если хотите остаться анонимным. Избегайте лишних вопросов.
\section{Ложная информация}
Для того, чтобы воспрепятствовать процессу установления вашей реальной личности вы можете специально давать ложную информацию. Например, вы можете сказать, что вы из Саратова, живя в Ростове-на-Дону. Поскольку тот, кто хочет установить вашу личность, не знает, где правда, а где ложь, то ваша ложь сильно усложняет ему задачу. Придумайте виртуального персонажа с выдуманными данными, от имени которого вы будете выступать.
\section{Лингвистический анализ}
Даже если вы не указываете в тексте личной информации, остается возможным установление вашей личности. Допущенные в тексте ошибки, сленг, редкоиспользуемые слова и многая другая информация может помочь в установлении личности автора с помощью лингвистического анализа. При составлении текстов, авторство которых вы хотите сохранить в секрете, используйте стиль, отличный от стиля, которым вы пользуетесь в реальной жизни.
