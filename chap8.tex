\chapter{Почему софт должен быть открытым?}
\section{Безопасность через неясность и принцип Керкгоффса}
\textbf{Безопасность через неясность (Security through obscurity)} --- уничижительное название принципа построения криптографических систем, согласно которому для обеспечения безопасности в тайне держатся особенности проектирования или реализации.\\
\textbf{Принцип Керкгоффса} --- принцип построения криптографических систем, сформулированный Огюстом Керкгоффсом в 1883 году, заключающийся в том, что в тайне должен держаться только определенный набор параметров алгоритма (ключ), а сам алгоритм должен быть открытым.
\section{Что такое Open Source}
\textbf{Open Source} --- программное обеспечение с открытым исходным кодом. Понятие Open Source не идентично понятию <<свободное ПО>> (хотя часто ПО принадлежит одновременно обоим категориям)  --- под свободным ПО подразумевают ПО, на которое действуют права на свободное использование, изучение, распространение и изменение, тогда как в случае с Open Source акцент делается на доступность исходных кодов.
\section{Почему проприетарное ПО бывает опасно}
\subsection{Обновление Windows с отключенной службой Windows Update}
24 августа 2007 года на компьютерах с отлюченной системой Windows Update (с отключенными автоматическими обновлениями) без ведома пользователя обновились некоторые файлы\cite{windowsupdate}. Теоретически это означает то, что Microsoft в любое время может исполнить любой код на любой Windows-машине (имеет бэкдур), даже если на них отключена служба автоматического обновления.
\subsection{Carrier IQ}
\textbf{Carrier IQ} --- компания, разрабатывающий софт, устанавливающийся производителем на многие мобильные устройства. Carrier IQ записывает данные с GPS-навигатора, состояние звонков, нажатия клавиш, интернет-трафик и многое другое, после чего в некоторых случаях отправляет эту информацию производителю устройства или на собственные сервера\cite{carrieriq1}\cite{carrieriq2}.
\subsection{Возможность получить IP адрес любого пользователя Skype}
После декомпиляции и деобфускации исходного кода Skype стало возможным узнать IP-адрес любого пользователя, зная только его ник. Инструкция о том, как это сделать, размещена здесь: \url{http://pastebin.com/LrW4NE2p}. IP-адрес можно узнать даже в течении трех дней после того, как пользователь вышел из сети. Это --- типичный пример излишней надежды на прицип <<безопасность через неясность>>, который привел к серьезным проблемам.
\subsection{Отправка данных о запускаемых приложениях в Windows 8}
\textbf{SmartScreen} --- новая технология, появившаяся в Windows 8 и включенная по умолчанию\cite{smartscreen}. SmartScreen работает так: когда вы скачиваете приложение из Интернета, то в Microsoft передается хеш файла, имя файла и сертификат (если присутствует)\cite{smartscreen_withinwindows}. При этом для передачи используется SSLv2, который не безопасен и злоумышленник может получить доступ к передаваемым данным\cite{smartscreen_nadim}.


